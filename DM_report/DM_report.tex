%%
%% Copyright 2007, 2008, 2009 Elsevier Ltd
%%
%% This file is part of the 'Elsarticle Bundle'.
%% ---------------------------------------------
%%
%% It may be distributed under the conditions of the LaTeX Project Public
%% License, either version 1.2 of this license or (at your option) any
%% later version.  The latest version of this license is in
%%    http://www.latex-project.org/lppl.txt
%% and version 1.2 or later is part of all distributions of LaTeX
%% version 1999/12/01 or later.
%%
%% This template has been modified by Philip Blakely for
%% local distribution to students on the MPhil for Scientific
%% Computing course run at the University of Cambridge.
%%

%% Template article for Elsevier's document class `elsarticle'
%% with numbered style bibliographic references
%% SP 2008/03/01
%%
%%
%%
%% $Id: elsarticle-template-num.tex 4 2009-10-24 08:22:58Z rishi $
%%
%%
\documentclass[final,3p,twocolumn]{elsarticle}

%% Use the option review to obtain double line spacing
%% \documentclass[preprint,review,12pt]{elsarticle}

%% Use the options 1p,twocolumn; 3p; 3p,twocolumn; 5p; or 5p,twocolumn
%% for a journal layout:
%% \documentclass[final,1p,times]{elsarticle}
%% \documentclass[final,1p,times,twocolumn]{elsarticle}
%% \documentclass[final,3p,times]{elsarticle}
%% \documentclass[final,3p,times,twocolumn]{elsarticle}
%% \documentclass[final,5p,times]{elsarticle}
%% \documentclass[final,5p,times,twocolumn]{elsarticle}

%% if you use PostScript figures in your article
%% use the graphics package for simple commands
%% \usepackage{graphics}
%% or use the graphicx package for more complicated commands
%% \usepackage{graphicx}
%% or use the epsfig package if you prefer to use the old commands
%% \usepackage{epsfig}

%% The amssymb package provides various useful mathematical symbols
\usepackage{amssymb}
\usepackage{amsmath}
%% The amsthm package provides extended theorem environments
%% \usepackage{amsthm}

%% The lineno packages adds line numbers. Start line numbering with
%% \begin{linenumbers}, end it with \end{linenumbers}. Or switch it on
%% for the whole article with \linenumbers after \end{frontmatter}.
%% \usepackage{lineno}

%% natbib.sty is loaded by default. However, natbib options can be
%% provided with \biboptions{...} command. Following options are
%% valid:

%%   round  -  round parentheses are used (default)
%%   square -  square brackets are used   [option]
%%   curly  -  curly braces are used      {option}
%%   angle  -  angle brackets are used    <option>
%%   semicolon  -  multiple citations separated by semi-colon
%%   colon  - same as semicolon, an earlier confusion
%%   comma  -  separated by comma
%%   numbers-  selects numerical citations
%%   super  -  numerical citations as superscripts
%%   sort   -  sorts multiple citations according to order in ref. list
%%   sort&compress   -  like sort, but also compresses numerical citations
%%   compress - compresses without sorting
%%
%% \biboptions{comma,round}

% \biboptions{}


\journal{MPhil in Scientific Computing}

\begin{document}

\begin{frontmatter}

%% Title, authors and addresses

%% use the tnoteref command within \title for footnotes;
%% use the tnotetext command for the associated footnote;
%% use the fnref command within \author or \address for footnotes;
%% use the fntext command for the associated footnote;
%% use the corref command within \author for corresponding author footnotes;
%% use the cortext command for the associated footnote;
%% use the ead command for the email address,
%% and the form \ead[url] for the home page:
%%
%% \title{Title\tnoteref{label1}}
%% \tnotetext[label1]{}
%% \author{Name\corref{cor1}\fnref{label2}}
%% \ead{email address}
%% \ead[url]{home page}
%% \fntext[label2]{}
%% \cortext[cor1]{}
%% \address{Address\fnref{label3}}
%% \fntext[label3]{}

\title{Improved Density Mixing Methods for \textit{Ab Initio} Calculations}

%% use optional labels to link authors explicitly to addresses:
%% \author[label1,label2]{<author name>}
%% \address[label1]{<address>}
%% \address[label2]{<address>}

\author{Nick Woods}

\address{Cavendish Laboratory, Department of Physics, J J Thomson
  Avenue, Cambridge. CB3 0HE}

\begin{abstract}
Density mixing $\dots$
\end{abstract}

\end{frontmatter}

%%
%% Start line numbering here if you want
%%
% \linenumbers

%% main text
\section{Introduction}

Density functional theory (DFT), and its implementation using the Kohn-Sham (KS) equations, forms a developed and widely successful foundation for performing first-principles calculations. The ability of DFT to extract accurate, quantum mechanical predictions efficiently has lead it to becoming the go-to method for calculations in materials science (ref), nuclear physics (ref), and more (ref). In its purest form, DFT is an exact (by construction) reformulation of quantum mechanics in the \textit{ground state}. It expresses the ground state energy variationally as a functional of a 3-dimensional electron density $\rho(\textbf{r})$ instead of the more complex $3N$-dimensional manybody wavefunction $\Psi( \{\textbf{r}_i \} )$. In practice, one solves the KS equations,
\begin{equation} \label{KSeq}
\hat{H}^{KS}[\rho] \phi_i(\textbf{r}) = \epsilon_i \phi_i(\textbf{r}),
\end{equation}
producing a set of single particle orbitals $\phi_i$ associated with the KS system (details in Section$.$ \ref{sec:DFT}). The electron density is then calculated from these single particle wavefunctions as such,
\begin{equation}\label{elecdens}
\rho(\textbf{r}) = \sum_{i \in \text{occupied}} |  \phi_i(\textbf{r}) |^2. lllllll
\end{equation}
The solutions $\phi_i$ depend on the input $\hat{H}^{KS}$ through the electron density $\rho$. This is therefore a \textit{non-linear} system of equations, where both Eq$.$ (\ref{KSeq}) and Eq$.$ (\ref{elecdens}) are required to be satisfied simoulteniously in order to solve KS DFT. Non-linear systems such as this can be solved using an iterative or self-consistent process. Namely, one seeks the fixed-point of the mapping KS mapping $F$ (denoting the process in Eq$.$ (\ref{KSeq}) and Eq$.$ (\ref{elecdens})),
\begin{equation}
F[\rho^{\text{in}}] = \rho^{\text{out}}, 
\end{equation}
which is solved by $\rho^* = \rho^{\text{in}} = \rho^{\text{out}}$ for a given external potential and electron number\footnote{$N = \int \rho(\textbf{r}) d\textbf{r}$.}. The idea of density mixing is then to find the fixed-point density by `optimally' combining $ \{ \rho_n^{\text{in}},\rho_n^{\text{out}} \}$ to drive $\rho^{\text{in}}_{n+1}$ toward $\rho^*$ -- here, $n$ denotes the iteration number. Optimising the self-consistent process by choosing the best available density mixing scheme is imperative for the success of KS DFT. Each update of the density requires diagonalisation of $\hat{H}^{KS}$ to re-solve the system and obtain an output density; this is a cumbersome task, and is ideally executed as little as possible. The aim of this project is to implement a proposed density mixing scheme (appropriate for planewaave DFT codes) by Marks and Luke [ref]. 


\section{Background Theory}


\subsection{Density Functional Theory}
\label{sec:DFT}

Density functional theoroy (DFT) is an exact (by construction) reformulation of quantum mechanics for systems in their ground state. Its foundation are the Hohenberg-Kohn theorems... implemented in practice using the Kohn-Sham (KS) equations (hereafter `KS DFT'). Now we know we need a technique for fixing the root (or fixed point) of the density in KS DFT, we can begin to study mathematically how one can achive that, what the best methods will be, and what the pitfalls are for DFT in particular. 


\subsection{The Iterative Process for Solving a Set of Nonlinear Equations}

Many problems in Physics take the form... and KS DFT, as discussed, is no different. The SCF operator 

%%\subsection{Iterative Techniques for Solving a Set of Nonlinear Equations}
\subsection{Fixed-Point Iterations}

Study, into to linear response regime, good initial guess. analysis of M

\subsection{Linear Mixing}

Improves on fixed point iterations, but suffers from slow convergence, or sloshing instabilities.

\subsection{Pulay Mixing}

Most popular method, method in CASTEP, DIIS.

\subsection{Broyden's Methods}

Another method in CASTEP, less popular, but important prerequisite for MSB1 and MSB2. 


\section{Multisecant Broyden Methods and Their Implementation in CASTEP}

This follows the work of Marks and Luke (ref) who propose two multisecant Broyden methods (based on the good and bad Broyden method) to improve mixing. Theory of the method... Implementation into CASTEP...

\section{Results}

First compare standard techniques already in CASTEP. Give background into how they are implemented, the parameters they use, how they compare. 
Now compare to the Marks and Luke method.

\section{Conclusions}


\section*{Acknowledgements}


%% The Appendices part is started with the command \appendix;
%% appendix sections are then done as normal sections
\appendix

\section{Derivation of Newton's Method}

\section{Derivation of Broyden's Methods}


%% References
%%
%% Following citation commands can be used in the body text:
%% Usage of \cite is as follows:
%%   \cite{key}         ==>>  [#]
%%   \cite[chap. 2]{key} ==>> [#, chap. 2]
%%

%% References with bibTeX database:

\bibliographystyle{elsarticle-num}
\bibliography{references.bib}

%% Authors are advised to submit their bibtex database files. They are
%% requested to list a bibtex style file in the manuscript if they do
%% not want to use elsarticle-num.bst.

%% References without bibTeX database:

% \begin{thebibliography}{00}

%% \bibitem must have the following form:
%%   \bibitem{key}...
%%

% \bibitem{}

% \end{thebibliography}


\end{document}

%%
%% End of file `elsarticle-template-num.tex'.
